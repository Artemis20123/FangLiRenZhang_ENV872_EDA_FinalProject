% Options for packages loaded elsewhere
\PassOptionsToPackage{unicode}{hyperref}
\PassOptionsToPackage{hyphens}{url}
%
\documentclass[
  12pt,
]{article}
\usepackage{amsmath,amssymb}
\usepackage{lmodern}
\usepackage{iftex}
\ifPDFTeX
  \usepackage[T1]{fontenc}
  \usepackage[utf8]{inputenc}
  \usepackage{textcomp} % provide euro and other symbols
\else % if luatex or xetex
  \usepackage{unicode-math}
  \defaultfontfeatures{Scale=MatchLowercase}
  \defaultfontfeatures[\rmfamily]{Ligatures=TeX,Scale=1}
  \setmainfont[]{Times New Roman}
\fi
% Use upquote if available, for straight quotes in verbatim environments
\IfFileExists{upquote.sty}{\usepackage{upquote}}{}
\IfFileExists{microtype.sty}{% use microtype if available
  \usepackage[]{microtype}
  \UseMicrotypeSet[protrusion]{basicmath} % disable protrusion for tt fonts
}{}
\makeatletter
\@ifundefined{KOMAClassName}{% if non-KOMA class
  \IfFileExists{parskip.sty}{%
    \usepackage{parskip}
  }{% else
    \setlength{\parindent}{0pt}
    \setlength{\parskip}{6pt plus 2pt minus 1pt}}
}{% if KOMA class
  \KOMAoptions{parskip=half}}
\makeatother
\usepackage{xcolor}
\usepackage[margin=2.54cm]{geometry}
\usepackage{longtable,booktabs,array}
\usepackage{calc} % for calculating minipage widths
% Correct order of tables after \paragraph or \subparagraph
\usepackage{etoolbox}
\makeatletter
\patchcmd\longtable{\par}{\if@noskipsec\mbox{}\fi\par}{}{}
\makeatother
% Allow footnotes in longtable head/foot
\IfFileExists{footnotehyper.sty}{\usepackage{footnotehyper}}{\usepackage{footnote}}
\makesavenoteenv{longtable}
\usepackage{graphicx}
\makeatletter
\def\maxwidth{\ifdim\Gin@nat@width>\linewidth\linewidth\else\Gin@nat@width\fi}
\def\maxheight{\ifdim\Gin@nat@height>\textheight\textheight\else\Gin@nat@height\fi}
\makeatother
% Scale images if necessary, so that they will not overflow the page
% margins by default, and it is still possible to overwrite the defaults
% using explicit options in \includegraphics[width, height, ...]{}
\setkeys{Gin}{width=\maxwidth,height=\maxheight,keepaspectratio}
% Set default figure placement to htbp
\makeatletter
\def\fps@figure{htbp}
\makeatother
\setlength{\emergencystretch}{3em} % prevent overfull lines
\providecommand{\tightlist}{%
  \setlength{\itemsep}{0pt}\setlength{\parskip}{0pt}}
\setcounter{secnumdepth}{5}
\newlength{\cslhangindent}
\setlength{\cslhangindent}{1.5em}
\newlength{\csllabelwidth}
\setlength{\csllabelwidth}{3em}
\newlength{\cslentryspacingunit} % times entry-spacing
\setlength{\cslentryspacingunit}{\parskip}
\newenvironment{CSLReferences}[2] % #1 hanging-ident, #2 entry spacing
 {% don't indent paragraphs
  \setlength{\parindent}{0pt}
  % turn on hanging indent if param 1 is 1
  \ifodd #1
  \let\oldpar\par
  \def\par{\hangindent=\cslhangindent\oldpar}
  \fi
  % set entry spacing
  \setlength{\parskip}{#2\cslentryspacingunit}
 }%
 {}
\usepackage{calc}
\newcommand{\CSLBlock}[1]{#1\hfill\break}
\newcommand{\CSLLeftMargin}[1]{\parbox[t]{\csllabelwidth}{#1}}
\newcommand{\CSLRightInline}[1]{\parbox[t]{\linewidth - \csllabelwidth}{#1}\break}
\newcommand{\CSLIndent}[1]{\hspace{\cslhangindent}#1}
\usepackage{booktabs}
\usepackage{longtable}
\usepackage{array}
\usepackage{multirow}
\usepackage{wrapfig}
\usepackage{float}
\usepackage{colortbl}
\usepackage{pdflscape}
\usepackage{tabu}
\usepackage{threeparttable}
\usepackage{threeparttablex}
\usepackage[normalem]{ulem}
\usepackage{makecell}
\usepackage{xcolor}
\ifLuaTeX
  \usepackage{selnolig}  % disable illegal ligatures
\fi
\IfFileExists{bookmark.sty}{\usepackage{bookmark}}{\usepackage{hyperref}}
\IfFileExists{xurl.sty}{\usepackage{xurl}}{} % add URL line breaks if available
\urlstyle{same} % disable monospaced font for URLs
\hypersetup{
  pdftitle={A Time Series Analysis and Visualization of PM2.5 Distribution in China},
  pdfauthor={Yixin Fang, Jiahuan Li, Yuxiang Ren, Jinglin Zhang},
  hidelinks,
  pdfcreator={LaTeX via pandoc}}

\title{A Time Series Analysis and Visualization of PM2.5 Distribution in
China}
\usepackage{etoolbox}
\makeatletter
\providecommand{\subtitle}[1]{% add subtitle to \maketitle
  \apptocmd{\@title}{\par {\large #1 \par}}{}{}
}
\makeatother
\subtitle{\url{https://github.com/Artemis20123/FangLiRenZhang_ENV872_EDA_FinalProject}}
\author{Yixin Fang, Jiahuan Li, Yuxiang Ren, Jinglin Zhang}
\date{}

\begin{document}
\maketitle

\newpage
\tableofcontents 
\newpage
\listoftables 
\newpage
\listoffigures 
\newpage

\hypertarget{rationale-and-research-questions}{%
\section{Rationale and Research
Questions}\label{rationale-and-research-questions}}

Numerous studies have shown that air pollution has significant impacts
on human health (\protect\hyperlink{ref-intro1}{Doe, Smith, \& Williams,
2020}), exposure to environments with air pollution can lead to a
variety of diseases, including cardiovascular disease, mental illness,
skin cancer, chronic respiratory diseases, and decreased lung function
Dockery \& Verrier (\protect\hyperlink{ref-intro3}{2005}). Among the
various pollutants in the environment, PM2.5 is considered a major
contributor to health problems (\protect\hyperlink{ref-intro4}{Pope III
\& Dockery, 2006}). Therefore, monitoring and controlling PM2.5 is
crucial for the sustainable development of society.

With a booming economy and rapid urbanization, China has been suffering
from severe air pollution problems. since China's economic reform in the
1980s, a large number of factories have been established rapidly in
China Compared with the growth of industry, China did not invest a lot
in environmental protection in the early stage. since 2006, China
updated its air pollution control measures and tried to control the
serious air pollution, In 2013, the central and eastern regions of China
experienced severe haze weather, leading the central government
implemented the Air Pollution Prevention and Control Action Plan to deal
with serious air pollution problems, since that, China has continuously
updated the ``Ambient Air Quality Standards'' and established a
nationwide monitoring network covering nearly 400 cities, which for the
first time included PM2.5 as a monitoring target
(\protect\hyperlink{ref-intro5}{Jin \& Zhang, 2016}). Due to the late
start of China's PM2.5 monitoring, the pollution status of PM2.5 in
China before 2013 was unclear. Wei et al.~(2021) filled this information
gap by interpreting MODIS Collection 6 MAIAC AOD to obtain the daily
average concentration of PM2.5 nationwide in China since 2000 for the
first time.

China currently has seven megacities with a population of over 10
million. These cities have frequent economic activities and high
population densities, understanding the changes in PM2.5 concentrations
in these areas and predicting their trends can provide important
references for China's air pollution prevention and control. This
project utilizes the data from Wei et al.~to visualize the monthly
average PM2.5 concentrations in these areas from 2000 to 2021, allowing
people to have a more intuitive understanding of the changes in local
air pollution, and predicting the changes in these areas for the next
three years. Our research question is:

\begin{enumerate}
\def\labelenumi{\arabic{enumi}.}
\tightlist
\item
  How has the PM2.5 concentration in the seven cities changed?
\item
  What might be the monthly average PM2.5 concentrations in these seven
  cities for the next 5 years?
\end{enumerate}

\newpage

\hypertarget{dataset-information}{%
\section{Dataset Information}\label{dataset-information}}

\hypertarget{data-sources}{%
\subsection{Data sources}\label{data-sources}}

This project relies on two primary datasets. The first dataset, China
\(PM_{2.5}\) pollution data, is available via the Zenodo data
repository\(^{[1]}\). This dataset provides information on the daily
concentration of particulate matter (\(PM_{2.5}\)) in the atmosphere
over China, as measured by satellite sensors. The dataset contains daily
average \(PM_{2.5}\) concentration values for a period from 2000 to
2021, covering the entirety of China's land area. The data is presented
in the form of raster images, with a spatial resolution of 0.01 degrees.
This high-resolution dataset allows for accurate identification of high
pollution cities and regions, and is an important resource for
understanding the severity and extent of air pollution in China.

The second dataset utilized in this project is a shapefile of Chinese
cities boundaries, provided by the statistical bureau of the Chinese
government. The shapefile includes Chinese city names, unique
identifiers, and geometry information for the city polygons. This
dataset enables the mapping of \(PM_{2.5}\) concentrations across
different Chinese cities and supports the time series analysis.
Together, these datasets form the foundation of our dashboard and other
relevant analysis, providing important insights into the spatial and
temporal distribution of \(PM_{2.5}\) pollution in China.

\hypertarget{wrangling-process}{%
\subsection{\texorpdfstring{\href{https://github.com/Artemis20123/FangLiRenZhang_ENV872_EDA_FinalProject/blob/main/Code/data\%20wrangling.Rmd}{Wrangling
process}}{Wrangling process}}\label{wrangling-process}}

The project involves two main wrangling processes. The first is to
convert raw NetCDF files to GeoTIFF files using Python code provided by
the original dataset creators. The original NetCDF files are not
included in our repository and can be downloaded from the open-access
Zenodo repository. The conversion code uses the \texttt{GDAL} package, a
powerful translator library for working with raster and vector
geospatial data formats. It reads the SDS data from the NetCDF files and
handles missing values, extracts longitude and latitude information, and
finally writes the output as GeoTIFF files. The use of \texttt{GDAL} in
this project ensures that the NetCDF files are accurately transformed
into GeoTIFF format, and that the resulting raster data can be easily
integrated with other geospatial datasets for further analysis and
visualization.

In the second data wrangling process, zonal statistics are computed to
aggregate the information of raster pixels within the city boundary
defined in the shapefile. The \texttt{terra} package's zonal function is
utilized to extract the statistics. The mean and sum \(PM_{2.5}\) values
of cities are generated as the outputs written in a CSV table. To
extract zonal statistics for each raster file, a loop function is used
to process each file in sequence. Two data frames are created to store
the sum and mean values separately. These data frames are then merged
together into a single data frame with two value columns. But only the
mean \(PM_{2.5}\) value is used in the following analysis.

Finally, all the CSV files containing separate daily \(PM_{2.5}\)
information are merged into one large CSV file. The daily records are
then summarized at the monthly level to facilitate dashboard
visualization and time series analysis.

\hypertarget{data-structure}{%
\subsection{\texorpdfstring{\href{https://github.com/Artemis20123/FangLiRenZhang_ENV872_EDA_FinalProject/blob/main/Data/Raw/Metadata.Rmd}{Data
structure}}{Data structure}}\label{data-structure}}

The \(PM_{2.5}\) pollution data used in the project is derived from the
ChinaHigh\(PM_{2.5}\) dataset, which provides a seamless 1 km
ground-level resolution of \(PM_{2.5}\) distribution in the context of
China. Since the original data (NetCDF files) are stored as images,
tables of the derived zonal statistics are shown below:

\begin{longtable}[]{@{}cccc@{}}
\caption{Monthly average \(PM_{2.5}\) concentrations of Chinese
cities}\tabularnewline
\toprule()
Variables & Class & Units & Ranges \\
\midrule()
\endfirsthead
\toprule()
Variables & Class & Units & Ranges \\
\midrule()
\endhead
year & numeric & T & {[}2000,2021{]} \\
month & numeric & T & {[}1,12{]} \\
city\_id & numeric & N/A & 1100 \textasciitilde{} 659001 \\
cityname & character & N/A & N/A \\
meanPM & numeric & \(µ\)g/m\(^3\) & {[}6.437489, 200.2201{]} \\
\bottomrule()
\end{longtable}

Additional metadata information can be found in the
\texttt{Metadata.Rmd} file located in the ``Data'' folder and ``Raw''
subfolder of the project. This file contains detailed information on the
data sources, data characteristics, variable definitions, as well as
information on the geospatial reference system and units.

\newpage

\hypertarget{exploratory-analysis}{%
\section{Exploratory Analysis}\label{exploratory-analysis}}

To explore the data, we wanted to create a visualization of the monthly
PM2.5 for the cities from 2000 to 2021. The data was first filtered to
include only the seven cities of interest. Then we grouped the data by
city, year, and month and summarized the mean PM2.5 value for each
month. A monthly PM2.5 line plot was created for each city using
ggplot2.

\begin{figure}
\centering
\includegraphics{LiFangRenZhang_ENV872_Project_files/figure-latex/TSA Visualization-1.pdf}
\caption{Time Series Visualization of the Seven Cities}
\end{figure}

Figure 1 showed that Beijing and Tianjin had the highest PM2.5 values
among the seven cities, and Shenzhen had the lowest values. Most of the
cities seemed to have a slow increase in PM2.5 from 2000 to 2006, then
stayed relatively constant from 2006 until 2014. In 2014, most cities
had a spike in PM2.5. From 2014, PM2.5 in all cities showed a decreasing
trend. To better understand the changes in PM2.5, we conducted
time-series analysis and predicted the monthly PM2.5 for the seven
cities from 2022 to 2026.

\newpage

\hypertarget{analysis}{%
\section{Analysis}\label{analysis}}

\hypertarget{part1-time-series-analysis}{%
\subsection{Part1: Time Series
Analysis}\label{part1-time-series-analysis}}

\hypertarget{how-has-the-pm2.5-concentration-in-the-seven-cities-changed}{%
\subsubsection{How has the PM2.5 concentration in the seven cities
changed?}\label{how-has-the-pm2.5-concentration-in-the-seven-cities-changed}}

Using the STL function, we decomposed the PM2.5 time series data of the
target cities into seasonal and trend components and performed a
comparative analysis. The PM2.5 values in these cities are generally
lower during the summer months (June, July, August, and September) and
higher during the winter months (November, December, January, and
February) (Figure 2). There are several possible factors contributing to
this seasonality: First, In the summer, temperatures are higher, leading
to more active air convection, which facilitates the dispersion and
dissipation of pollutants in the air. Secondly, there tends to be more
rainfall in the summer, which can wash PM2.5 particles from the
atmosphere to the ground, reducing their measured values
(\protect\hyperlink{ref-rain}{Pu, Zhao, Zhang, \& Ma, 2011}). Thirdly,
in the winter, heating demand increases, and in some areas, coal
combustion remains the primary method of providing heat
(\protect\hyperlink{ref-heating}{Liang et al., 2015}). This leads to
increased emissions of coal-related pollutants, causing a rise in PM2.5
levels.

\begin{figure}
\centering
\includegraphics{LiFangRenZhang_ENV872_Project_files/figure-latex/seasonal figure-1.pdf}
\caption{Seasonality in PM2.5 in 7 Chinese megacities}
\end{figure}

The trend results for the target cities are consistent with those
observed in the exploratory analysis (Figure 3). Notably, around 2014
marked a turning point when PM2.5 levels in major cities began to
decrease gradually. This shift is likely attributed to the Chinese
government's implementation of the Air Pollution Prevention and Control
Action Plan (\protect\hyperlink{ref-plan}{Cai et al., 2017}). It
required that, by 2017, the annual average PM2.5 concentration in cities
at or above the prefectural level should be reduced by more than 10\%.
In key regions such as Beijing-Tianjin-Hebei, the Yangtze River Delta,
and the Pearl River Delta, the plan aimed for reductions of 25\%, 20\%,
and 15\%, respectively.

\begin{figure}
\centering
\includegraphics{LiFangRenZhang_ENV872_Project_files/figure-latex/trend figure-1.pdf}
\caption{Trends in PM2.5 in Chinese megacities}
\end{figure}

\hypertarget{what-might-be-the-monthly-average-pm2.5-concentrations-in-these-seven-cities-for-the-next-three-years}{%
\subsubsection{What might be the monthly average PM2.5 concentrations in
these seven cities for the next three
years?}\label{what-might-be-the-monthly-average-pm2.5-concentrations-in-these-seven-cities-for-the-next-three-years}}

In order to obtain better forecasting results, we conducted a predictive
accuracy test on multiple time series models based on Beijing's data
from 2000 to 2020 (using actual 2021 data as a benchmark for
comparison). These models include:

\begin{enumerate}
\def\labelenumi{\arabic{enumi}.}
\tightlist
\item
  Arithmetic Mean Model (Mean): This model assumes that future
  observations will equal current observations, i.e., the predicted
  future value equals the current value, making the model a constant
  average value model.
\item
  Seasonal Naive Model (SNAIVE): It assumes that future observations
  will equal the most recent observations from the same season. The
  model focuses solely on historical data within the same season.
\item
  Seasonal Autoregressive Integrated Moving Average Model (SARIMA):
  SARIMA is a widely-used seasonal time series model that builds upon
  the ARIMA model by incorporating seasonal variations in the time
  series data.
\item
  Seasonal Simple Exponential Smoothing (SSES) Model: This is a time
  series forecasting method based on weighted averages.
\end{enumerate}

In these four models tested, the SSES model had the smallest Root Mean
Square Error (RMSE) and the best forecasting capability (Table 1).
However, considering that the forecasting target is for the next five
years, both SSES and SNAIVE models predict the same values for the
first, second, third, fourth, and fifth years when forecasting multiple
years. As a result, we combined the results of SSES and SNAIVE with the
SARIMA model, creating two new models:

\begin{table}

\caption{\label{tab:accuacy_table1}Forecast Accuracy for Seasonal Data}
\centering
\begin{tabular}[t]{l|r|r|r|r|r}
\hline
  & ME & RMSE & MAE & MPE & MAPE\\
\hline
MEAN & -28.1059 & 31.6898 & 29.2560 & -119.0707 & 120.7650\\
\hline
SNAIVE & -2.4987 & 13.4377 & 9.3796 & -17.3618 & 28.6924\\
\hline
SARIMA & -4.3333 & 14.5863 & 12.2154 & -32.3573 & 45.6127\\
\hline
\cellcolor{gray!6}{SSES} & \cellcolor{gray!6}{4.8008} & \cellcolor{gray!6}{12.3008} & \cellcolor{gray!6}{7.7936} & \cellcolor{gray!6}{5.2446} & \cellcolor{gray!6}{19.6416}\\
\hline
\end{tabular}
\end{table}

\begin{enumerate}
\def\labelenumi{\arabic{enumi}.}
\setcounter{enumi}{4}
\tightlist
\item
  SNAIVE\_SARIMA. The average of SNAIVE and SARIMA.
\item
  SSES\_SARIMA. The average of SSES and SARIMA.
\end{enumerate}

Although the accuracy results show that the RMSE of SSES\_SARIMA is the
second smallest, its predictive accuracy is still slightly lower than
that of SSES (Table 2). However, considering the forecasting objective
and the comparison with other models, we ultimately chose to use the
SSES\_SARIMA model for predicting PM2.5 levels in the target cities.

\begin{table}

\caption{\label{tab:accuacy_table2}Forecast Accuracy for Seasonal Data}
\centering
\begin{tabular}[t]{l|r|r|r|r|r}
\hline
  & ME & RMSE & MAE & MPE & MAPE\\
\hline
MEAN & -28.105858 & 31.68980 & 29.256003 & -119.070730 & 120.76499\\
\hline
SNAIVE & -2.498701 & 13.43770 & 9.379570 & -17.361823 & 28.69245\\
\hline
SARIMA & -4.333324 & 14.58628 & 12.215420 & -32.357291 & 45.61268\\
\hline
\cellcolor{gray!6}{SSES} & \cellcolor{gray!6}{4.800801} & \cellcolor{gray!6}{12.30076} & \cellcolor{gray!6}{7.793640} & \cellcolor{gray!6}{5.244604} & \cellcolor{gray!6}{19.64161}\\
\hline
SNAIVE\_SARIMA & -3.416013 & 13.00090 & 9.967983 & -24.859557 & 35.26887\\
\hline
SSES\_SARIMA & 0.233739 & 12.50184 & 9.574127 & -13.556343 & 31.19364\\
\hline
\end{tabular}
\end{table}

The final forecast results of Chinese megacities are shown in Figure 4.

\begin{figure}
\centering
\includegraphics{LiFangRenZhang_ENV872_Project_files/figure-latex/forecast figure-1.pdf}
\caption{PM2.5 forecast in Chinese megacities}
\end{figure}

\newpage

\hypertarget{dashboard}{%
\subsection{Dashboard:}\label{dashboard}}

To better visualize and allow users to interact with the data, we
created a dashboard using Shiny. The dashboard consisted of three side
panels:``PM2.5 National Distribution'', ``Time Series Visualization by
City'', ``PM2.5 Prediction by City'' and three corresponding tab panels:
``Map'', ``TSA'', and ``Prediction''.

The ``PM2.5 National Distribution'' allows users to drag the map in the
``Map'' tab and explore the PM2.5 distribution pattern in the country.
The ``Time Series Visualization by City'' panel has a dropdown box that
allows users to select the monthly PM2.5 visualization of each city. The
results are displayed in the ``TSA'' tab. Finally, the ``PM2.5
Prediction by City'' panel has a dropdown box and slider. Users can
select the city and the time range to view the monthly PM2.5 prediction
in the ``Prediction'' tab.

\newpage

\hypertarget{summary-and-conclusions}{%
\section{Summary and Conclusions}\label{summary-and-conclusions}}

We visualized the monthly average concentration of PM2.5 in Chinese
cities from 2000 to 2021 and analyzed the historical trends and
predicted the concentration changes for the next 5 years in megacities.
Our results show that the PM2.5 concentration increased from 2000 to
2013, with higher concentrations in the eastern and northen regions
compared to the western and southern regions. Since 2014, PM2.5
concentration in China has been decreasing. The concentration during
autumn and winter was higher than that during spring and summer.
Combining the SRIMA and SSES models resulted in the most accurate
prediction. Our prediction suggest that at the end of 2027, the highest
and lowest pm2.5 in Beijing, .

\newpage

\hypertarget{references}{%
\section*{References}\label{references}}
\addcontentsline{toc}{section}{References}

\hypertarget{refs}{}
\begin{CSLReferences}{1}{0}
\leavevmode\vadjust pre{\hypertarget{ref-intro2}{}}%
Brunekreef, B., \& Holgate, S. T. (2002). Air pollution and health.
\emph{The Lancet}, \emph{360}(9341), 1233--1242.

\leavevmode\vadjust pre{\hypertarget{ref-plan}{}}%
Cai, S., Wang, Y., Zhao, B., Wang, S., Chang, X., \& Hao, J. (2017). The
impact of the {``air pollution prevention and control action plan''} on
PM2. 5 concentrations in jing-jin-ji region during 2012--2020.
\emph{Science of the Total Environment}, \emph{580}.

\leavevmode\vadjust pre{\hypertarget{ref-intro3}{}}%
Dockery, L.-G., D. W., \& Verrier, R. L. (2005). Association of air
pollution with increased incidence of ventricular tachyarrhythmias
recorded by implanted cardioverter defibrillators. \emph{Environmental
Health Perspectives}, \emph{113}(6), 670--674.

\leavevmode\vadjust pre{\hypertarget{ref-intro1}{}}%
Doe, J., Smith, D., JaneFowler, \& Williams, M. L. (2020). Global air
quality, past present and future: An introduction. \emph{Philosophical
Transactions of the Royal Society A}, \emph{378}(2183), 20190323.

\leavevmode\vadjust pre{\hypertarget{ref-intro5}{}}%
Jin, A., Y., \& Zhang, S. (2016). Air pollution control policies in
china: A retrospective and prospects. \emph{International Journal of
Environmental Research and Public Health}, \emph{13}(12), 1219.

\leavevmode\vadjust pre{\hypertarget{ref-heating}{}}%
Liang, X., Zou, T., Guo, B., Li, S., Zhang, H., Zhang, S., \ldots{}
Chen, S. X. (2015). Assessing beijing's PM2. 5 pollution: Severity,
weather impact, APEC and winter heating. \emph{Proceedings of the Royal
Society A: Mathematical, Physical and Engineering Sciences}, \emph{471}.

\leavevmode\vadjust pre{\hypertarget{ref-intro4}{}}%
Pope III, C. A., \& Dockery, D. W. (2006). Health effects of fine
particulate air pollution: Lines that connect. \emph{Journal of the Air
and Waste Management Association}, \emph{56}(6), 709--742.

\leavevmode\vadjust pre{\hypertarget{ref-rain}{}}%
Pu, W., Zhao, X., Zhang, X., \& Ma, Z. (2011). Effect of meteorological
factors on PM2. 5 during july to september of beijing. \emph{Procedia
Earth and Planetary Science}, \emph{2}, 272--277.

\end{CSLReferences}

\end{document}
